\section{Assignment 16.8}
\textbf{Assignment Description}
\begin{figure}[H]
	\includegraphics[width = \linewidth]{assignment168}
\end{figure}
The reason why serial equivalence requires that once a transaction has released a lock on an object, it is not allowed to obtain any more locks is the following:\\
If a transaction locks an object after already having released it once, other transactions could potentially try to access and manipulate the object. This could result in the transaction ending up with a wrong result e.g. if a bank transaction is not serial equivalent, there could be to much money or to little in an account after the transaction has ended, because the transactions have been working on different versions of an object.\\\\
A non serial equivalent interleaving of the transactions \textit{T} and \textit{U} could be:\\
\textit{T}: x = read(i)
\textit{U}: write(i,55)
\textit{U}: write(j,66)
\textit{T}: write(j,44)
