\section{Assignment 16.3}

\textbf{(I) Strict:} \\
Before (U) accessing a resource, a transaction must wait for all previous transactions (T), accessing the same resource, to either commit or abort. \\Transaction T will lock j and i. This is Strict and Transaction T will not unlock the locked resources before all reads and writes are done.\\\\ Transaction U will commit after (write(k, 66). 

\begin{center}
	\begin{tabular}{| l | l |}
		\hline
		Transaction T & Transaction U \\ \hline
		x:=read(j) &  \\ \hline
		y:=read(i) &  \\ \hline
		 & x:=read(k) \\ \hline
		write(j, 44) &   \\ \hline
		write(i, 33) &  \\ \hline
		COMMIT &  \\ \hline
		 & write(i, 55)  \\ \hline
		 & y:=read(j)  \\ \hline
		 & write(k, 66)  \\
		\hline
	\end{tabular}
\end{center}

\textbf{(II) Not strict, no cascading aborts:} \\
Not strict since U access i before T commits, but there’s no cascading abort since there can be no dirty reads.

\begin{center}
	\begin{tabular}{| l | l |}
		\hline
		Transaction T & Transaction U \\ \hline
		x:=read(j) &  \\ \hline
		y:=read(i) &  \\ \hline
		 & x:=read(k) \\ \hline
		write(j, 44) &   \\ \hline
		write(i, 33) &  \\ \hline
		 & write(i, 55)  \\ \hline
		COMMIT &  \\ \hline
		 & y:=read(j)  \\ \hline
		 & write(k, 66) \\ 
		\hline
	\end{tabular}
\end{center}


\textbf{(III) With Cascading aborts:} \\
U is reading before T commits. Therefore, a cascading abort can happen after U tries to read j it is not committed yet. 

\begin{center}
	\begin{tabular}{| l | l |}
		\hline
		Transaction T & Transaction U \\ \hline
		x:=read(j) &  \\ \hline
		y:=read(i) &  \\ \hline
		 & x:=read(k) \\ \hline
		write(j, 44) &   \\ \hline
		write(i, 33) & \\ \hline
		 & write(i, 55) \\ \hline
		 & y:=read(j) \\ \hline
		COMMIT &  \\ \hline
		 & write(k, 66)  \\
		\hline
	\end{tabular}
\end{center}